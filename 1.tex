%%%%%%%%%%%%%%%%%%%%%%%%
%                      %
%       CURSUL 1       %
%                      %
%%%%%%%%%%%%%%%%%%%%%%%%
\part{}

\section{Șiruri de funcții reale}

\paragraph{}
Fie $(f_{n})_{n \in \mathbb{Z}}.$

$f_{n}: D \subseteq \mathbb{R} \rightarrow \mathbb{R}$

\paragraph{Definiția 1.}
Spunem că șirul de funcții $(f_{n})_{n \in \mathbb{Z}}$ converge simplu pe
mulțimea $A \subseteq D$ dacă $\forall \, \, x \in A$, șirul
$(f_{n})_{n \in \mathbb{Z}} \subseteq \mathbb{R}$ este convergent.

\paragraph{NOTAȚIE}
$\displaystyle\lim_{n \rightarrow \infty} f_{n}(x) = f(x)$,
$\, f_n \, \autorightarrow{$s$}{$n\rightarrow\infty$} f$
pe mulțimea $A$, $f: A \rightarrow \mathbb{R}$

\paragraph{Definiția 2.}

Spunem că șirul de funcții $(f _{n} ) _{ n\in\mathbb{Z} }$ converge uniform
pe mulțimea $A \subseteq D$ dacă $\exists \, f:A \rightarrow \mathbb{R}$ cu
proprietatea că $\forall \, \varepsilon>0, \exists \, n_{\varepsilon} \in
\mathbb{N}$ astfel încât $\lvert f_{n}(x) - f(x) \rvert < \varepsilon,
\forall \, n \geq n_{\varepsilon}, \forall \, x \in A.$

\paragraph{NOTAȚIE}
$f_{n} \autorightarrow{$u$}{$n\rightarrow\infty$} f$ pe mulțimea $A$.

\paragraph{Observație}
$f_{n} \autorightarrow{$u$}{$n\rightarrow\infty$} f$ pe mulțimea $A
\subseteq D \Rightarrow f_n$ $\autorightarrow{$s$}{$n\rightarrow\infty$} f$
pe mulțimea $A$.

\hspace{187pt}
$\nLeftarrow$

\subsection{Criteriul practic de convergență uniformă pentru un șir de
funcții}
Următoarele afirmații sunt echivalente:
\begin{enumerate}[label=\emph{\alph*})]
    \item $f_{n} \autorightarrow{$u$}{$n\rightarrow\infty$} f$ pe mulțimea
	    $A \subseteq D$.
    \item $\displaystyle\lim_{n \rightarrow \infty}(\sup_{x \in A} \lvert
	    f_{n}(x) - f(x) \rvert) = 0$
\end{enumerate}

\subsection{Criteriul lui Cauchy pentru limite de funcții}
Următoarele afirmații sunt echivalente:
\begin{enumerate}[label=\emph{\alph*})]
	\item $(f_{n})_{n \in \mathbb{N}}$ converge uniform pe mulțimea
		$A \subseteq D$
	\item $\forall \, \varepsilon>0$, $\exists \, n_{\varepsilon} \in
		\mathbb{N}$ astfel încât $\vert f_{n}(x) - f_{m}(x) \rvert <
		\varepsilon$, $\forall \, n, m \geq n_{\varepsilon}$,
		$\forall \, x \in A$
\end{enumerate}

\subsection{Teorema lui Weierstrass}
Considerăm un șir de funcții $(f_{n})_{n \in \mathbb{N}}$, o funcție $f:A
\subseteq D \rightarrow \mathbb{R}$ astfel încât $f_{n}
\autorightarrow{$u$}{$n \rightarrow \infty$} f$ pe mulțimea $A$. Dacă
$\exists \, x_{0} \in A$ astfel încât $f_{n}$ este continuă în $x_{0}$,
$\forall \, n \in \mathbb{N}$, atunci $f$ este continuă în $x_{0}$.

\subsection{Teorema Stone-Weierstrass}
Pentru orice funcție continuă $f:[a,b] \rightarrow \mathbb{R}$, există un
șir de funcții polinomiale $(p_{n})_{n \in \mathbb{N}}$ astfel încât $p_{n}
\autorightarrow{$u$}{$n \rightarrow \infty$} f$ pe mulțimea $[a, b]$.

\subsection{Teorema lui Dini}
Considerăm un șir monoton de funcții continue $(f_{n})_{n \in \mathbb{N}}$,
$f_{n}: [a, b] \rightarrow \mathbb{R}$ ($f_{n} \leq f_{n+1}$, $\forall \, n
\in \mathbb{N}$) sau ($f_{n} \geq f_{n+1}$, $\forall \, n \in \mathbb{N}$)
și o funcție continuă $f: [a, b] \rightarrow \mathbb{R}$ astfel încât $f_{n}
\autorightarrow{$s$}{$n \rightarrow \infty$} f$ pe mulțimea $[a, b]$.

Atunci $f_{n} \autorightarrow{$u$}{$n \rightarrow \infty$} f$ pe mulțimea
$[a, b]$.

\subsection{Teorema lui Polya}
Considerăm un șir de funcții continue și monotone $(f_{n})_{n \in
\mathbb{N}}$, $f_{n}:[a,b] \rightarrow \mathbb{R}$ și o funcție continuă
$f:[a,b] \rightarrow \mathbb{R}$ astfel încât
$f_{n} \autorightarrow{$s$}{$n \rightarrow \infty$} f$ pe mulțimea $[a, b]$.

Atunci $f_{n} \autorightarrow{$u$}{$n \rightarrow \infty$} f$ pe mulțimea
$[a, b]$.

\section{Serii de funcții reale}

$(f_{n})_{n \in \mathbb{N}}$; $f_{n}: D \subseteq \mathbb{R} \rightarrow
\mathbb{R}$ \\
$(f_{n})_{n \in \mathbb{N}}$ $\rightarrow$ $(s_{n})_{n \in \mathbb{N}}$ \\
$s_{n}:D \subseteq \mathbb{R} \rightarrow \mathbb{R}$ \\
$s_{n}(x) = f_{0}(x) + f_{1}(x) + \mathellipsis + f_{n}(x)$

\paragraph*{Definiția 1.}

Perechea de șiruri de funcții ($(f_{n})_{n \in \mathbb{N}}$, $(s_{n})_{n \in
\mathbb{N}}$) se numește seria de funcții atașată șirului de funcții
$(f_{n})_{n \in \mathbb{N}}$ și se notează
$\displaystyle\sum_{n=0}^{+\infty} f_{n}$.

\paragraph*{Definiția 2.}
\begin{enumerate}[label=\emph{\alph*})]
    \item Spunem că seria de funcții $\displaystyle\sum_{n =
	    0}^{+\infty}f_{n}$ este simplu convergentă pe mulțimea $A
	    \subseteq D$ dacă șirul de funcții $(s_{n})_{n \in \mathbb{N}}$
	    converge simplu pe mulțimea $A$.
    \item Spunem că seria de funcții $\displaystyle\sum_{n =
	    0}^{+\infty}f_{n}$ este absolut convergentă pe mulțimea $A
	    \subseteq D$ dacă seria de funcții $\displaystyle\sum_{n =
	    0}^{+\infty} \lvert f_{n} \rvert$ este simplu convergentă pe
	    mulțimea $A$.
    \item Spunem că seria de funcții $\displaystyle\sum_{n =
	    0}^{+\infty}f_{n}$ este uniform convergentă pe mulțimea $A
	    \subseteq D$ dacă șirul de funcții $(s_{n})_{n \in \mathbb{N}}$
	    converge uniform pe mulțimea $A$.
\end{enumerate}

\paragraph*{Observații}
\begin{enumerate}[label=\emph{\alph*})]
    \item Din seria de funcții $\displaystyle\sum_{n = 0}^{+\infty}f_{n}$
	    este absolut convergentă pe mulțimea $A \subseteq D$, atunci ea
	    este simplu convergentă pe mulțimea $A \subseteq D$.
    \item Dacă seria de funcții $\displaystyle\sum_{n = 0}^{+\infty}f_{n}$
	    este uniform convergentă pe mulțimea $A \subseteq D$, atunci ea
	    este simplu convergentă pe mulțimea $A \subseteq D$
\end{enumerate}

\subsection{Criteriul lui Cauchy pentru serii de funcții}
Următoarele afirmații sunt echivalente:
\begin{enumerate}[label=\emph{\alph*})]
    \item seria de funcții $\displaystyle\sum_{n = 0}^{+\infty}f_{n}$ este
	    uniform convergentă pe mulțimea $A \subseteq D$.
    \item $\forall \, \varepsilon > 0$, $\exists \, n_{\varepsilon} \in
	    \mathbb{N}$ astfel încât $\lvert f_{n}(x) + f_{n+1}(x) +
	    \mathellipsis + f_{n+p}(x) \rvert < \varepsilon$, $\forall \, n
	    \geq n_{\varepsilon}$, $\forall \, p \in \mathbb{N}$, $\forall
	    \, x \in A$.
\end{enumerate}

\subsection{Criteriul lui Weierstrass pentru serii de funcții}
Fie $(f_{n})_{n \in \mathbb{N}}$ un șir de funcții, $f_{n}:D \subseteq
\mathbb{R} \rightarrow \mathbb{R}$, $(a_{n})_{n \in \mathbb{N}} \subseteq
\mathbb{R}_{+}$, $A \subseteq D$ astfel încât $\lvert f_{n}(x) \rvert \leq
a_{n}$, $\forall \, x \in A$, $\forall \, n \in \mathbb{N}$. Dacă seria de
numere reale $\displaystyle\sum_{n = 0}^{+\infty}a_{n}$ este convergentă,
atunci seria de funcții $\displaystyle\sum_{n = 0}^{+\infty}f_{n}$ este
uniform convergentă pe mulțimea $A$.

\paragraph{Demonstrație}
$\displaystyle\sum_{n = 0}^{+\infty}a_{n}$ este convergentă $\Rightarrow$ 

$\Rightarrow$$\forall \, \varepsilon > 0$, $\exists \, n_{\varepsilon} \in
\mathbb{N}$ astfel încât $\lvert a_{n} + a_{n+1} + \mathellipsis + a_{n+p}
\rvert < \varepsilon$, $\forall \, n \geq n_{\varepsilon}$, $\forall \, p
\in \mathbb{N}$ $\Rightarrow$

$\Rightarrow$$\forall \, \varepsilon > 0$, $\exists \, n_{\varepsilon} \in
\mathbb{N}$ astfel încât $a_{n} + a_{n+1} + \mathellipsis + a_{n+p} <
\varepsilon$, $\forall \, n \geq n_{\varepsilon}$, $\forall \, p \in
\mathbb{N}$ \ding{172} \\[5pt]

$\lvert f_{n}(x)+f_{n+1}(x)+\mathellipsis+f_{n+p}(x) \rvert \leq \lvert
f_{n}(x) \rvert + \lvert f_{n+1}(x) \rvert + \mathellipsis + \lvert
f_{n+p}(x) \rvert \leq a_{n}(x)+a_{n+1}(x)+\mathellipsis+a_{n+p}(x)$,
$\forall \, x \in A$, $\forall \, n,p \in \mathbb{N}$. \ding{173} \\[15pt]

Din \ding{172} și \ding{173} $\Rightarrow$ $\forall \, \varepsilon > 0$,
$\exists \, n_{\varepsilon} \in \mathbb{N}$ astfel încât \\

$\lvert f_{n}(x)+f_{n+1}(x)+\mathellipsis+f_{n+p}(x) \rvert < \varepsilon$,
$\forall \, n \geq n_{\varepsilon}$, $\forall \, p \in \mathbb{N}$, $\forall
\, x \in A$. \\

$\xRightarrow{Cauchy}$ seria de funcții
$\displaystyle\sum_{n=0}^{+\infty}f_{n}$ este uniform și absolut convergentă
pe $A$.
