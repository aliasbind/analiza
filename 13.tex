%%%%%%%%%%%%%%%%%%%%%%%%
%                      %
%       CURSUL 13      %
%                      %
%%%%%%%%%%%%%%%%%%%%%%%%
\part{}
\paragraph{Propoziția 1.}
Fie $f \in \mathcal{R}([a,b])$ și considerăm funcțiile $F,G:[a,b] \rightarrow \mathbb{R}$ definite prin
$F(x) = \displaystyle\int_{a}^{x} f(t) \, dt, G(x) = \displaystyle\int_{x}^{b}f(t) \, dt$. Atunci $F,G$
sunt funcții continue pe $[a,b]$. \\[5pt]
Dacă $f$ este continuă în $x_{0} \in [a,b]$, $F, G$ sunt derivabile în $x_{0}$ și $F'(x_{0}) = f(x_{0}), G'(x_{0})=-f(x_{0})$. \\[5pt]
Dacă $f$ este continuă pe $[a,b]$, $F, G$ sunt derivabile pe $[a,b]$ și $F'(x)=f(x)$, $G'(x) = -f(x)$, $\forall x \in [a,b]$.

\paragraph{Corolar}
Orice funcție continuă $f:\underset{\text{interval}}{I} \subseteq \mathbb{R} \rightarrow \mathbb{R}$ admite primitive.

\paragraph{Construcția unei primitive:}
$a \in I$ se fixează. $F:I \rightarrow \mathbb{R}$, $F(x)=\displaystyle\int_{a}^{b} f(t) \, dt$.

\paragraph{Propoziția 2.}
Fie $f:\underset{\text{interval}}{I} \subseteq \mathbb{R} \rightarrow \mathbb{R}$ o funcție continuă și $g,h:J \subseteq \mathbb{R} \rightarrow I \subseteq \mathbb{R}$
două funcții derivabile pe $J$. Atunci funcția $F:J \rightarrow \mathbb{R}$, $F(x) = \displaystyle\int_{g(x)}^{f(x)} f(t) \, dt$ este derivabilă pe $J$
și \\ $F'(x) = f(h(x) \cdot h'(x)) - f(g(x) \cdot g'(x))$, $\forall x \in J$.

\subsubsection{Formula Leibniz-Newton}
Fie $f \in \mathcal{R}([a,b])$ astfel încât $f$ admite primitive. Atunci $\displaystyle\int_{a}^{b} f(x) \, dx = \Bigl.F(x) \Bigr|_{a}^{b} =
F(b) - F(a)$, unde $F$ este o primitivă a lui $f$.

\subsubsection{Formula de integrare prin părți}
Fie $f, g:[a,b] \rightarrow \mathbb{R}$ două funcții derivabile pe $[a,b]$ astfel încât $f', g' \in \mathcal{R}([a,b])$.
Atunci \\ $\displaystyle\int_{a}^{b} f(x)g'(x) \, dx = \Bigl.f(x)g(x) \Bigr|_{a}^{b} - \displaystyle\int_{a}^{b} f'(x)g(x) \, dx$.

\subsubsection{Schimbarea de variabilă pentru integralele definite}
Fie $f \in \mathcal{R}([a,b])$ și $f:[\alpha, \beta] \rightarrow [a,b]$ o funcție bijectivă cu $\varphi(\alpha) = a$,
$\varphi(\beta) = b$, derivabilă cu $\varphi'$ funcție continuă. Atunci $(f \circ \varphi) \cdot \varphi' \in \mathcal{R}([a,b])$
și $\displaystyle\int_{\alpha}^{\beta} f(\varphi(x)) \cdot \varphi'(t) \, dt = \displaystyle\int_{a = \varphi(\alpha)}^{b = \varphi(\beta)} f(x) \, dx$.

\paragraph{Teorema 1.}
Fie $f, g \in \mathcal{R}([a,b])$.
\begin{enumerate}[label=\emph{\alph*})]
    \item Dacă $f(x) \geq 0$, $\forall x \in [a,b]$, atunci $\displaystyle\int_{a}^{b} f(x) \, dx \geq 0$.
    \item Dacă $f(x) \geq g(x)$, $\forall x \in [a,b]$, atunci $\displaystyle\int_{a}^{b} f(x) \, dx \geq \displaystyle\int_{a}^{b} g(x) \, dx$.
    \item Are loc inegalitatea $m(b-a) \leq \displaystyle\int_{a}^{b} f(x) \, dx \leq M(b-a)$, unde $m = \displaystyle\inf_{t \in [a,b]}f(t)$, $M = \displaystyle\sup_{t \in [a,b]}f(t)$.
    \item Fie $f:[a,b] \rightarrow \mathbb{R}$ o funcție continuă astfel încât $f(x) \geq 0$, $\forall x \in [a,b]$ și $\displaystyle\int_{a}^{b} f(x) \, dx=0$.
        Atunci $f(x) = 0$, $\forall x \in [a,b]$.
    \item Fie $f:[a,b] \rightarrow \mathbb{R}$ o funcție continuă astfel încât $f(x) \geq 0$, $\forall x \in [a,b]$ și $\exists \, x_{0} \in [a,b]$
        astfel încât $f(x_{0}) > 0$. Atunci $\displaystyle\int_{a}^{b} f(x) \, dx \geq 0$.
\end{enumerate}

\subsubsection{Prima teoremă de medie pentru integralele definite}
Fie $f, g \in \mathcal{R}([a,b])$ astfel încât $f$ are proprietatea lui Darboux pe $[a,b]$ și $g \geq 0$. Atunci $\exists \, c \in (a, b)$
astfel încât $\displaystyle\int_{a}^{b} f(x) g(x) \, dx = f(c) \cdot \displaystyle\int_{a}^{b} g(x) \, dx$.

\paragraph{Corolar}
Dacă $f:[a,b] \rightarrow \mathbb{R}$ este funcție continuă, $\exists \, c \in (a,b)$ astfel încât \\
$\displaystyle\int_{a}^{b} f(x) \, dx = f(c) \cdot \displaystyle\int_{a}^{b} 1 \, dx = f(c) \, (b-a)$.

\subsubsection{A doua teoremă de medie pentru integralele definite}
Fie $f, g \in \mathcal{R}([a,b])$ astfel încât $g$ este funcție monotonă. Atunci $\exists \, c \in [a,b]$
astfel încât \\ $\displaystyle\int_{a}^{b} f(x) \, g(x) \, dx = g(a) \cdot \displaystyle\int_{a}^{c} f(x) \, dx + 
g(b) \cdot \displaystyle\int_{c}^{b} f(x) \, dx$.

\paragraph{Corolar}
Fie $f:[a,b] \rightarrow \mathbb{R}$ o funcție continuă și $g:[a.b] \rightarrow \mathbb{R}_{+}$.
\begin{enumerate}[label=\emph{\alph*})]
    \item Dacă $g$ este crescătoare, $\exists \, c \in [a,b]$ astfel încât $\displaystyle\int_{a}^{b} f(x) \, g(x) \, dx =
        g(b) \cdot \displaystyle\int_{b}^{c} f(x) \, dx$.
    \item Dacă $g$ este descrescătoare, $\exists \, c \in [a,b]$ astfel încât $\displaystyle\int_{a}^{b} f(x) \, g(x) \, dx =
        g(a) \cdot \displaystyle\int_{a}^{c} f(x) \, dx$.
\end{enumerate}

\subsection{Integrala improprie}
$f: I \rightarrow \mathbb{R}$ \\
$I = [a,b) \vee (a, b] \vee (a, b) \vee (-\infty, a] \vee (-\infty, a) \vee (a, +\infty) \vee [a, +\infty) \vee \mathbb{R}$ ($\vee$ = sau).

\paragraph{Definiția 1.}
Funcția $f: I \rightarrow \mathbb{R}$ se numește local integrabilă pe $I$ dacă $\forall \, \alpha < \beta \in I$, $\bigl. f \bigr|_{[\alpha, \beta]} \in \mathcal{R}([\alpha, \beta]$. 

\paragraph{NOTAȚIE}
$\mathcal{R}_{loc}(I)$ $^{\ushortdw{def}}$ $\{ f: I \rightarrow \mathbb{R}$ $|$ $f$ este local integrabilă pe $I \}$. \\[10pt]
Exemple de funcții local integrabile:
\begin{enumerate}[label=\emph{\arabic*})]
    \item Orice funcție monotonă $f: I \rightarrow \mathbb{R}$ este local integrabilă pe $I$.
    \item Orice funcție continuă $f: I \rightarrow \mathbb{R}$ este local integrabilă pe $I$.
\end{enumerate}
\rule{450pt}{1pt}\\[5pt]
$f:[a,b) \rightarrow \mathbb{R}$, $\displaystyle\int_{a}^{b-0} f(x) \, dx$.

\paragraph{Definiția 2.}
Fie $f \in \mathcal{R}_{loc}([a,b])$. Integrala improprie $\displaystyle\int_{a}^{b-0} f(x) \, dx$ se numește convergentă dacă
$\exists \, \displaystyle\lim_{\substack{x \rightarrow b \\ x < b}} \displaystyle\int_{a}^{x} f(t) \, dt \in \mathbb{R}$. \\[8pt]
$f:(a, b) \rightarrow \mathbb{R}$,
$\exists \, \displaystyle\lim_{\substack{x \rightarrow a \\ x > a \\ y \rightarrow b \\ y < b}} \displaystyle\int_{x}^{y} f(t) \, dt \in \mathbb{R}$. \\[8pt]
$f:[a, +\infty) \rightarrow \mathbb{R}$,
$\exists \, \displaystyle\lim_{x \rightarrow \infty} \displaystyle\int_{a}^{x} f(t) \, dt \in \mathbb{R}$. \\[8pt]
Integrala improprie $\displaystyle\int_{a}^{b-0}f(x) \, dx$ se numește divergentă dacă nu este convergentă. \\
Integrala improprie $\displaystyle\int_{a}^{b-0}f(x) \, dx$ se numește absolut convergentă dacă integrala $\displaystyle\int_{a}^{b-0} \left| f(x) \right| \, dx$ este convergentă. \\

\subsubsection{Criteriul lui Cauchy pentru integralele improprii}
Fie $f \in \mathcal{R}_{loc}([a,b])$.
\begin{enumerate}[label=\emph{\alph*})]
    \item $\displaystyle\int_{a}^{b-0} f(x) \, dx$ este convergentă $\Leftrightarrow \forall \, \varepsilon > 0, \exists \, t_{\varepsilon} \in [a,b)$ astfel încât
        $\left| \displaystyle\int_{x}^{y} f(t) \, dt \right| < \varepsilon$, $\forall t_{\varepsilon} < x < y < b$.
    \item $\displaystyle\int_{a}^{b-0} f(x) \, dx$ este absolut convergentă $\Leftrightarrow \forall \, \varepsilon > 0, \exists \, t_{\varepsilon} \in [a,b)$ astfel încât
        $\displaystyle\int_{x}^{y} \left| f(t) \right| \, dt < \varepsilon$, $\forall t_{\varepsilon} < x < y < b$.
\end{enumerate}

\paragraph{Teorema 1.}
Dacă $\displaystyle\int_{a}^{b-0} f(x) \, dx$ este absolut convergentă atunci este convergentă. \\
Dacă $\displaystyle\int_{a}^{b-0} f(x) \, dx$ este absolut convergentă $\underset{\text{Cauchy (b)}}{\xLongrightarrow{\text{Criteriul lui}}}$
$\forall \, \varepsilon > 0$, $\exists \, t_{\varepsilon} \in [a, b)$ astfel încât $\displaystyle\int_{x}^{y} \left| f(t) \right| dt < \varepsilon$,
$\forall \, t_{\varepsilon} < x < y < b$ \ding{172}. \\[6pt]
Știm că $\left| \displaystyle\int_{x}^{y} f(t) \, dt \right| \leq \int_{x}^{y} \left| f(t) \right| dt$, $\forall x < y$ \ding{173}. \\[5pt]
Din \ding{172} și \ding{173} $\Rightarrow \left| \displaystyle\int_{x}^{y} f(t) \, dt \right| < \varepsilon$, $\forall \, t_{\varepsilon} < x < y < b$
$\underset{\text{Cauchy (b)}}{\xLongrightarrow{\text{Criteriul lui}}} \displaystyle\int_{a}^{b-0} f(x) \, dx$ este convergentă. \\[8pt]
Reciproca teoremei este falsă.

