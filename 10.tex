%%%%%%%%%%%%%%%%%%%%%%%%
%                      %
%       CURSUL 10      %
%                      %
%%%%%%%%%%%%%%%%%%%%%%%%
\part{}
\section{Aplicații biliniare și continue}

\paragraph{Definiția 1.}
\begin{enumerate}[label=\emph{\alph*})]
    \item $T: \mathbb{R}^{k} \times \mathbb{R}^{k} \rightarrow \mathbb{R}^{p}$ se numește biliniară dacă
        \begin{itemize}
            \item $T(\alpha u+\beta v, w) = \alpha T(u,w) + \beta T(v,w)$
            \item $T(u, \alpha v+\beta w) = \alpha T(u,v) + \beta T(u,w)$
        \end{itemize}
        $\forall \alpha$, $\beta \in \mathbb{R}$, $\forall u, v, w \in \mathbb{R}^{p}$
    \item O aplicație biliniară $T:\mathbb{R}^{k} \times \mathbb{R}^{k} \rightarrow \mathbb{R}^{p}$ se numește simetrică dacă
        $T(u,v) = T(v,u)$, $\forall u, v \in \mathbb{R}^{k}$
\end{enumerate}

\paragraph{Definiția 2.}
Fie $T:\mathbb{R}^{k} \times \mathbb{R}^{k} \rightarrow \mathbb{R}$ o aplicație biliniară.

\begin{enumerate}[label=\emph{\alph*})]
    \item $T$ se numește pozitivă ($T \geq 0$) dacă $T(u,u) \geq 0$, $\forall u \in \mathbb{R}^{k}$
    \item $T$ se numește strict pozitivă ($T > 0$) dacă $T(u,u) > 0$, $\forall u \in \mathbb{R}^{k} \setminus \{0\}$
    \item $T$ se numește negativă ($T \leq 0$) dacă $T(u,u) \leq 0$, $\forall u \in \mathbb{R}^{k}$
    \item $T$ se numește strict negativă ($T < 0$) dacă $T(u,u) < 0$, $\forall u \in \mathbb{R}^{k} \setminus \{0\}$
\end{enumerate}

\paragraph{Propoziția 1.}
Aplicația $T:\mathbb{R}^{k} \times \mathbb{R}^{k} \rightarrow \mathbb{R}^{p}$ este biliniară dacă și numai dacă
$\exists$ $\{ a_{ij} \vert 1 \leq i, j \leq k \} \subseteq \mathbb{R}^{p}$ astfel încât
$T((x_{1}, x_{2}, \mathellipsis, x_{k}), (y_{1}, y_{2}, \mathellipsis, y_{k})) = \displaystyle\sum_{i,j=1}^{k} x_{i}y_{j}a_{ij},
\forall (x_{1}, \mathellipsis, x_{k}), (y_{1}, \mathellipsis, y_{k}) \in \mathbb{R}^{k}$. \\
În plus, $T$ este aplicație biliniară simetrică dacă și numai dacă $a_{ij} = a_{ji}$, $\forall 1 \leq i, j \leq k$.

\paragraph{Propoziția 2.}
Fie $T:\mathbb{R}^{k} \times \mathbb{R}^{k} \rightarrow \mathbb{R}$ o aplicație biliniară simetrică și $A=(a_{ij})_{1 \leq i, j \leq k} \in \mathcal{M}_{k}(\mathbb{R})$
matricea asociată lui $T$.

\begin{enumerate}[label=\emph{\alph*})]
    \item $T>0 \Leftrightarrow \forall 1 \leq l \leq k, det(a_{ij})_{1 \leq i, j \leq l}$
    \item $T<0 \Leftrightarrow \forall 1 \leq l \leq k, (-1)^{l} det(a_{ij})_{1 \leq i, j \leq l} > 0$
\end{enumerate}

\paragraph{Propoziția 3.}
Orice aplicație biliniară $T:\mathbb{R}^{k} \times \mathbb{R}^{k} \rightarrow \mathbb{R}^{p}$ este continuă.

\paragraph{NOTAȚIE}
$\mathcal{L}(\mathbb{R}^{k}, \mathbb{R}^{k}, \mathbb{R}^{p})$ $^{\ushortdw{def}}$
$\{ T:\mathbb{R}^{k} \times \mathbb{R}^{k} \rightarrow \mathbb{R}^{p}$ $\vert$ $T$ aplicație liniară și continuă $\}$

\section{Derivate parțiale de ordinul doi. \\ Diferențiala de ordinul doi}
$f:D \subseteq \mathbb{R}^{k} \rightarrow \mathbb{R}^{p}$ \\
$f(x_{1}, x_{2}, \mathellipsis, x_{k}) = (f_{1}(x_{1}, \mathellipsis, x_{k}), \mathellipsis, f_{p}(x_{1}, \mathellipsis, x_{k}))$ \\
$f = (f_{1}, f_{2}, \mathellipsis, f_{p})$ \\
$f_{1}, f_{2}, \mathellipsis, f_{p}:D \subseteq \mathbb{R}^{k} \rightarrow \mathbb{R}$

\paragraph{Definiția 1.}
Spunem că $f$ admite derivată parțială de ordinul doi, în raport cu variabilele $x_{i}$ și $x_{j}$ în punctul $x_{0} \in D \cap D'$ dacă
$\exists$ $V \in \mathcal{V}(x_{0})$ astfel încât $f$ admite derivată parțială în raport cu variabila $x_{j}$ pe $D \cap V$ și
$\displaystyle\frac{\partial f}{\partial x_{j}}$ admite derivată parțială în raport cu variabila $x_{i}$ în $x_{0}$.

\paragraph{NOTAȚIE}
$\displaystyle\frac{\partial^{2} f}{\partial x_{i} \partial x_{j}}(x_{0}) = 
\displaystyle\frac{\partial}{\partial x_{i}}\left(\displaystyle\frac{\partial f}{\partial x_{j}}\right)(x_{0})$

\paragraph{Observație}
$\exists$ $\displaystyle\frac{\partial^{2} f}{\partial x_{i} \partial x_{j}}(x_{0}) \in \mathbb{R}^{p} \Leftrightarrow \exists$
$\displaystyle\frac{\partial^{2} f_{1}}{\partial x_{i} \partial x_{j}}(x_{0})$, $\mathellipsis, 
\displaystyle\frac{\partial^{2} f_{p}}{\partial x_{i} \partial x_{j}}(x_{0}) \in \mathbb{R}$. \\
$\displaystyle\frac{\partial^{2} f}{\partial x_{i} \partial x_{j}}(x_{0}) = 
\left(\displaystyle\frac{\partial^{2} f_{1}}{\partial x_{i} \partial x_{j}}(x_{0}) \text{,} \mathellipsis, 
\displaystyle\frac{\partial^{2} f_{p}}{\partial x_{i} \partial x_{j}}(x_{0}) \right)$.

\paragraph{Definiția 2.}
Spunem că $f$ este diferențiabilă de două ori în $x_{0} \in D \cap D'$ dacă $\exists$ $V \in \mathcal{V}(x_{0})$ astfel încât
$f$ este diferențiabilă pe $V \cap D$ și $df$ este diferențiabilă în $x_{0}$.

\paragraph{NOTAȚIE}
$d^{2}f_{x_{0}} = d(df)_{x_{0}}$, $d^{2}f_{x_{0}}: \mathbb{R}^{k} \times \mathbb{R}^{k} \rightarrow \mathbb{R}^{p}$ aplicație biliniară și continuă.

\subsection{Teorema lui Schwarz}
Dacă $f$ este diferențiabilă de două ori în $x_{0}$, atunci $f$ admite toate derivatele parțiale de ordinul doi în $x_{0}$.
În plus, au loc următoarele egalități:
\begin{enumerate}[label=\emph{\alph*})]
    \item $\displaystyle\frac{\partial^{2} f}{\partial x_{i} \partial x_{j}}(x_{0}) = \displaystyle\frac{\partial^{2} f}{\partial x_{j} \partial x_{i}}(x_{0})$,
        $\forall 1 \leq i, j \leq k$
    \item $d^{2}f_{x_{0}}((x_{1}, \mathellipsis, x_{k}), (y_{1}, \mathellipsis, y_{k})) = \displaystyle\sum_{i,j=1}^{k}x_{i}y_{j} 
        \displaystyle\frac{\partial^{2} f}{\partial x_{i} \partial x_{j}}(x_{0})$, $\forall (x_{1}, \mathellipsis, x_{k}), (y_{1}, \mathellipsis, y_{k}) \in \mathbb{R}^{k}$.
\end{enumerate}

\paragraph{Observații}
\begin{enumerate}[label=\emph{\arabic*})]
    \item Dacă $f:D \subseteq \mathbb{R}^{k} \rightarrow \mathbb{R}^{p}$ este diferențiabilă de două ori în $x_{0} \in D \cap D'$, 
        $d^{2}f:\mathbb{R}^{k} \times \mathbb{R}^{k} \rightarrow \mathbb{R}^{p}$ este aplicație biliniară, continuă și simetrică.
    \item Dacă $f:D \subseteq \mathbb{R}^{k} \rightarrow \mathbb{R}$ este diferențiabilă de două ori în $x_{0} \in D \cap D'$, 
        $d^{2}f:\mathbb{R}^{k} \times \mathbb{R}^{k} \rightarrow \mathbb{R}$ este aplicație biliniară, continuă și simetrică.
\end{enumerate}
I se asociază matricea $A = \left( \displaystyle\frac{\partial^{2} f}{\partial x_{i} \partial x_{j}} \right)_{1 \leq i, j \leq k} \in \mathcal{M}_{k}(\mathbb{R})$
care se numește hessiana funcției $f$ în $x_{0}$.

\subsubsection{Corolare la teorema lui Schwarz}
\begin{enumerate}[label=\emph{\arabic*})]
    \item Dacă $f$ nu admite cel puțin o derivată parțială de ordin doi în $x_{0}$, atunci $f$ nu este diferențiabilă de două ori în $x_{0}$.
    \item Dacă $f$ admite derivatele parțiale de ordinul doi în $x_{0}$ și $\exists$ $1 \leq i, j \leq k$ astfel încât
        $\displaystyle\frac{\partial^{2} f}{\partial x_{i} \partial x_{j}}(x_{0}) \neq \displaystyle\frac{\partial^{2} f}{\partial x_{j} \partial x_{i}}(x_{0})$
        atunci $f$ nu este diferențiabilă de două ori în $x_{0}$.
\end{enumerate}

\subsection{Teorema lui Young}
Fie $f:D \subseteq \mathbb{R}^{k} \rightarrow \mathbb{R}^{p}$, $x_{0} \in \mathring{D}, V \in \mathcal{V}(x_{0})$ astfel încât
$V \subseteq D$ și $f$ admite toate derivatele parțiale de ordin doi pe $V$ și acestea sunt funcții continue în $x_{0}$. Atunci
$f$ este diferențiabilă de două ori în $x_{0}$.

\paragraph{Corolar}
Fie $f:D \subseteq \mathbb{R}^{k} \rightarrow \mathbb{R}^{p}$, $G = \mathring{G} \subseteq D$ astfel încât
$f$ admite toate derivatele parțiale de ordin doi pe $G$ și acestea sunt funcții continue pe $G$.
Atunci $f$ este diferențiabilă de două ori pe $G$.
